\begin{table}[h]
\centering
\begin{tabular}{|l|l|l|l|}
\hline
\textbf{n}          & \textbf{n!}           & \textbf{n**2}        & \textbf{sqrt(n)}       \\ \hline
\textbf{0}          & \textbf{1}            & \textbf{0}           & \textbf{0.00}             \\ \hline
\textbf{1}          & \textbf{1}            & \textbf{1}           & \textbf{1.00}             \\ \hline
\textbf{2}          & \textbf{2}            & \textbf{4}           & \textbf{1.41}          \\ \hline
\textit{\textbf{3}} & \textit{\textbf{6}}   & \textit{\textbf{9}}  & \textit{\textbf{1.73}} \\ \hline
\textit{\textbf{4}} & \textit{\textbf{24}}  & \textit{\textbf{16}} & \textit{\textbf{2.00}} \\ \hline
\textit{\textbf{5}} & \textit{\textbf{120}} & \textit{\textbf{25}} & \textit{\textbf{2.23}} \\ \hline
\end{tabular}
\label{tab: functions}
\caption{Podstawowe funkcje matematyczne}
\end{table}
